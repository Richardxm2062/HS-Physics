\documentclass{article}
\input{../../../../../LaTex/preamble/preamble_article.tex}

\begin{comment}
    重复语句
   
\end{comment}


\title{高中物理8月初升高班课}
\author{教师:\quad 马祥芸}

\begin{document}
\maketitle
\tableofcontents
\newpage
\zihao{-4}

\section{李星语}
\begin{center}
    \resizebox{0.8\textwidth}{!}{
        \begin{tabular}{|c|c|c|c|c|c|c|}
            \hline
            姓名   & 性别 & 高中   & 年级   & 教材  & 学科 & 成绩              \\
            \hline
            李星语 & 女  & 杨家坪中学 & 2027 & 人教版 & 物理 & \\
            \hline
        \end{tabular}
    }
\end{center}

\subsection{学情分析}

上课有时候比较专注,有时候只听但没有思考,记忆能力尚可,在具体情景中对公式的使用非常缓慢。有比较强的自尊心

班课测试成绩: 30

\vspace{2em}

\section{彭正纭}

\begin{center}
    \resizebox{0.8\textwidth}{!}{
        \begin{tabular}{|c|c|c|c|c|c|c|}
            \hline
            姓名  & 性别 & 高中       & 年级   & 教材  & 学科 & 成绩   \\
            \hline
            彭正纭 & 男  & 育才科学城 & 2027 & 人教版 & 物理 & \\
            \hline
        \end{tabular}
    }
\end{center}

\subsection{学情分析}
对新知识的接受能力还不错,就是喜欢讲话,开小差(吃东西,讲话)。学习能力没那么踏实

班课测试成绩: 缺考 知识点全忘 重新上一对一
\vspace{2em}

\section{张嘉益}

\begin{center}
    \resizebox{0.8\textwidth}{!}{
        \begin{tabular}{|c|c|c|c|c|c|c|}
            \hline
            姓名  & 性别 & 高中 & 年级   & 教材  & 学科 & 成绩   \\
            \hline
            张嘉益 & 男  & 育才科学城 & 2027 & 人教版 & 物理 &  \\
            \hline
        \end{tabular}
    }
\end{center}

\subsection{学情分析}
学习能力很不错,就是经常喊难,笔记做的很少,计算过程也很少不规范。

班课测试成绩: 40+

\end{document}
