\documentclass{article}
\input{../../../../../LaTex/preamble/preamble_article.tex}

\begin{comment}
    重复语句
   
\end{comment}


\title{高中物理7月高一升高二班课}
\author{教师:\quad 马祥芸}

\begin{document}
\maketitle
\tableofcontents
\newpage
\zihao{-4}

\section{胡皖子开}
\begin{center}
    \resizebox{0.8\textwidth}{!}{
        \begin{tabular}{|c|c|c|c|c|c|c|}
            \hline
            姓名   & 性别 & 高中   & 年级   & 教材  & 学科 & 成绩              \\
            \hline
            胡皖子开 & 男  & 育才成功 & 2026 & 人教版 & 物理 & 初45\% \, 高40 \% \\
            \hline
        \end{tabular}
    }
\end{center}

\subsection{学情分析}

初中yh上高级一对一,指标到校走成功育才学校
做笔记没有那么勤快,上课有时候精神没那么集中,对基本简单的物理概念把握不好,作业反馈比较真实

班课测试成绩: 60

\vspace{2em}

\section{魏彤桦}

\begin{center}
    \resizebox{0.8\textwidth}{!}{
        \begin{tabular}{|c|c|c|c|c|c|c|}
            \hline
            姓名  & 性别 & 高中       & 年级   & 教材  & 学科 & 成绩   \\
            \hline
            魏彤桦 & 男  & 杨家坪中学名校班 & 2026 & 人教版 & 物理 & 65\% \\
            \hline
        \end{tabular}
    }
\end{center}

\subsection{学情分析}
上课很认真,积极动笔计算,新知识吸收也比较好

班课测试成绩: 75
\vspace{2em}

\section{龚君昊}

\begin{center}
    \resizebox{0.8\textwidth}{!}{
        \begin{tabular}{|c|c|c|c|c|c|c|}
            \hline
            姓名  & 性别 & 高中 & 年级   & 教材  & 学科 & 成绩   \\
            \hline
            龚君昊 & 男  & 育才 & 2026 & 人教版 & 物理 & 50\% \\
            \hline
        \end{tabular}
    }
\end{center}

\subsection{学情分析}
原清北班后一直下滑。上课听讲很仔细,做笔记也很认真,但是对方法的吸收比较差。做题建议不能很好吸收,积极纠正

班课测试成绩: 40+

\vspace{2em}

\section{张歆妤}

\begin{center}
    \resizebox{0.8\textwidth}{!}{
        \begin{tabular}{|c|c|c|c|c|c|c|}
            \hline
            姓名  & 性别 & 高中 & 年级   & 教材  & 学科 & 成绩 \\
            \hline
            张馨月 & 女  & 育才 & 2026 & 人教版 & 物理 &    \\
            \hline
        \end{tabular}
    }
\end{center}

\subsection{学情分析}
上课有时注意力没那么集中,容易打瞌睡,新知识的接受较差,基础较差

班课测试成绩: 40+

\vspace{2em}

\section{邱云霄}

\begin{center}
    \resizebox{0.8\textwidth}{!}{
        \begin{tabular}{|c|c|c|c|c|c|c|}
            \hline
            姓名  & 性别 & 高中 & 年级   & 教材  & 学科 & 成绩 \\
            \hline
            邱云霄 & 男  & 育才 & 2026 & 人教版 & 物理 & 60\%  \\
            \hline
        \end{tabular}
    }
\end{center}

\subsection{学情分析}
上课不喜欢抬头,作业反馈通常比较好,但真实性不够高。整体学习态度没那么扎实

班课测试成绩: 60+





\end{document}
