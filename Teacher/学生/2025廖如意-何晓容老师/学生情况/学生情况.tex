\documentclass{article}
\input{../../../../../LaTex/preamble/preamble_article.tex}

\begin{comment}
    重复语句
   
\end{comment}


\title{高中物理}
\author{学生:\quad 廖如意 \quad 教师:\quad 马祥芸}

\begin{document}
    \maketitle
    \tableofcontents
    \newpage
    \zihao{-4}

    \section{学生情况}
    \begin{center}
        \resizebox{0.8\textwidth}{!}{
        \begin{tabular}{|c|c|c|c|c|c|c|}
            \hline
            姓名 & 性别 & 高中 & 年级 & 教材 & 学科 & 成绩 \\
            \hline
            廖如意 & 女 & 育才 & 2025 & 人教版 & 物理 & 60\% \\
            \hline
        \end{tabular} 
        } 
    \end{center}
        
    \section{学情分析}

    有一定基础,想攻克较难题目,以及易错的基础题

    8月暑假课程(12次课):
    各个板块基础题,易错题
    较难题目
    电磁感应单双杆模型


\end{document}
