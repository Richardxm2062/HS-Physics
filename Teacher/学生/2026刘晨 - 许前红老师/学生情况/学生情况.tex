\documentclass{article}
\input{../../../../../LaTex/preamble/preamble_article.tex}

\begin{comment}
    重复语句
   
\end{comment}


\title{高中物理}
\author{学生:\quad 刘晨 \quad 教师:\quad 马祥芸}

\begin{document}
    \maketitle
    \tableofcontents
    \newpage
    \zihao{-4}

    \section{学生情况}
    \begin{center}
        \resizebox{0.8\textwidth}{!}{
        \begin{tabular}{|c|c|c|c|c|c|c|}
            \hline
            姓名 & 性别 & 高中 & 年级 & 教材 & 学科 & 成绩 \\
            \hline
            刘晨 & 男 & 天星桥中学 & 2026 & 人教版 & 物理 & 30\% \\
            \hline
        \end{tabular} 
        } 
    \end{center}
        
    \section{学情分析}
    6月课程:
    续讲运动学

    7月8月课程:
    高一内容重讲

    学生精神通常不会很好,需要多沟通交流,积极寻找反馈

    基础薄弱,在讲解部分需要减少数学思维.讲义上的题要多做,尽可能让学生自己多动笔

    作业适当降低难度

    学生动笔积极性很差! 课后大概率也不会复习,所以上课的时候要多默写公式


\end{document}
